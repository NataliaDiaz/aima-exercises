\documentclass{article}
\usepackage{fleqn}
\usepackage{epsf}
%\usepackage[dvips]{color}
\usepackage{aima3e}


\section{Introduction}
%%%% 2.2: Good Behavior: The Concept of Rationality (4 exercises, 2 labelled)
%%%% ========================================================================

\begin{uexercise}
Suppose that the performance measure is concerned with just the first \(T\) time steps of the environment and ignores everything thereafter.
Show that a rational agent's action may depend not just on the state of the environment but also on the time step it has reached.
\end{uexercise} 
% id=2.1 section=2.2

\begin{exercise}[vacuum-rationality-exercise]%
Let us examine the rationality of various vacuum-cleaner agent functions.
\begin{enumerate}
\item Show that the simple vacuum-cleaner agent function described in
\tabref{vacuum-agent-function-table} is indeed rational
under the assumptions listed on \pgref{vacuum-rationality-page}.
\item Describe a rational agent function for the case in which
each movement costs one point. Does the corresponding agent program require internal state?
\item Discuss possible agent designs for
the cases in which clean squares can become dirty and the geography
of the environment is unknown. Does it make sense for the agent to
learn from its experience in these cases? If so, what should it learn?
If not, why not?
\end{enumerate}
\end{exercise} 
\end{document}







